

\section{Related work}


In the BIP framework, DFinder~\cite{dfinder10} is a dedicated tool for invariant generation and deadlock-freedom detection.
%
DFinder computes the system invariant in a compositional manner:
it first computes a component invariant over-approximating the reachable states of each component
and then computes an interaction invariant over-approximating the global reachable states.
The system invariant is then the conjunction of all component invariants and the interaction invariant.
%
Though being scalable for large system models,
DFinder does not handle system models with data transfer,
which hampers the practical application of DFinder and of the BIP framework,
since data transfer is necessary and common in the design of real-life systems (e.g. message passing).
%
Besides, it is not clear in DFinder how to refine the abstraction automatically
when the inferred invariant fails to justify the property.

A compositional encoding of BIP into nuXmv
 and an efficient instantiation of Explicit Scheduler Symbolic Thread (ESST) framework for BIP have been presented in~\cite{atva15}.
 The encoding into nuXmv allows one to exploit the state-of-the-art verification techniques to verify BIP models.
%
The ESST based technique encodes the components as preemptive threads with predefined primitive functions and
 utilizes a dedicated stateful BIP scheduler to orchestrate the abstract reachability analysis of the components.
 The scheduler interacts with components via primitive functions, and also respects BIP operational semantics.
 Moreover, partial order reduction techniques are applied in the scheduler to reduce the states of scheduler.

In~\cite{tgc15}, we present a lazy predicate abstraction technique for BIP, and
 we also propose a novel reduction technique called simultaneous set reduction,
 which is further combined with lazy abstraction to reduce the search space of the abstract reachability analysis.
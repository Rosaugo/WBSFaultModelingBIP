
\section{Architecture-based rigorous system design}
\label{sec:rsd}


In this section, we present the BIP model with multiparty synchronization and data transfer.
%
%we denote by $\varstate{}[x := e]$ the substitution of variable $x$ by expression $e$ in valuation $\varstate{}$.
%
A BIP model is a parallel composition of a set of components.
A BIP component is formally defined as an automaton extended with linear integer arithmetic as follows.

\begin{definition} [BIP component]
\label{component-def}
Given a finite set of variables $\bipvar{}$, a BIP component is defined as a tuple
 $\comp{} = \mktuple{\bipvar{}, \locs{}, \ports{}, \edges{}, \ell}$, where
\begin{inparaenum}
\item $\locs{}$ is a finite set of control locations;
\item $\ports{}$ is a finite set of communication ports;
\item $\edges{} \subseteq \locs{} \times \ports{} \times \bexp{\bipvar{}} \times \expr{\bipvar{}} \times \locs{}$
 is a finite set of transition edges extended with guards in $\bexp{\bipvar{}}$
 and operations in $\expr{\bipvar{}}$;
\item $\ell \in \locs{}$ is an initial control location.
\end{inparaenum}
\end{definition}


Transition edges in a component are labeled by ports, which form the interface of the component.
 We assume that, from each control location,
 every pair of outgoing transitions have different ports,
 and the ports of different components are disjoint.
 In other words, transitions with the same ports in the component are not enabled simultaneously.
 Given a component violating such assumptions,
 one can easily transform it into the required form by renaming the ports,
 while retaining the BIP expressiveness power.
 To ease the presentation, we denote in the sequel the id of the unique component
 where port $p$ is defined by $\portid{p}$.

%Given a tuple of component types $\overline{\comptype{}} = \mktuple{\comptype{1},\ldots, \comptype{k}}$,
%and a tuple of natural numbers $\overline{n} = \mktuple{n_{1}, \ldots, n_{k}}$,
%where $n_{i}$ represents the number of instantiations of $\comptype{i}$,
%

We denote by $\components = \{\comp{i} ~|~ i\in [1,n]\}$ a set of components.
%instantiated from $\overline{\comptype{}}$.
%where $\comptype{i}[j]$
%$\comptype{i}[j] = \mktuple{\bipvar{i}[j] ,\locs{i}[j] ,\ports{i}[j], \edges{i}[j], \ell_{i}[j] }$
%is the $j^{th}$ instantiation of component type $\comptype{i}$.
%
%Given a set of comoponents $\components$,
In \BIP, coordinations of components are specified by using interactions.
%and in \BIP an interaction is represented as a finite set of ports.

\begin{definition} [Interaction]
\label{interaction}
An interaction for $\components $ is a tuple $\gamma = \mktuple{g, \mathcal{P}, f } $,
where $g \in \bexp{\bipvar{}}$, $f\in \expr{\bipvar{}}$ and
$\mathcal{P} \subseteq \bigcup_{i=1}^n \ports{i}$, $\mathcal{P} \neq \emptyset$, and
for all $i \in [1, n]$, $|\mathcal{P} \cap \ports{i}| \le 1$.
\end{definition}

Intuitively, an interaction defines a guarded multiparty synchronization with data transfer:
when the guard $g$ of an interaction $\mathcal{P}$ is enabled,
then the data transfer specified by $f$ can be executed, and
after that the transitions labelled by the ports in $\gamma$ can be taken simultaneously.
%
We denote by $\Gamma$ a finite set of interactions.
A \BIP model is constructed by composing a number of components with interactions.

\begin{definition} [\BIP Model]
A \BIP model $\bipmodel$ is a tuple $\mktuple{\components,\Gamma}$,
 where $\components$ is a finite set of components,
 and $\Gamma$ is a finite set of interactions for $\components$.
\end{definition}

We do not take priority into account in this paper, as in the previous work~\cite{dfinder10,tgc15},
 since adding priority will not introduce any errors.
 If a model without priority is safe, then after adding priority constraints it remains safe.
 We use a simple mutual exclusion protocol to illustrate \BIP.

%\input{ticket-bip}

A state of a \BIP model is a tuple $c = \langle \compstate{1}, \ldots, \compstate{n} \rangle $,
 where for all $i \in [1,n] $, $l_i \in \locs{i} $ and $\mathbf{V}_i $ is a valuation of $\bipvar{i}$.
%
%We denote by $c(i,j), i\in[1,k]$ the state $\mktuple{l_{i}^{j}, s_{i}^{j}}$ of component $\comptype{i}[j]$.
%$c_0=\mktuple{\mktuple{l_1,s_1},\ldots, \ldots, \mktuple{l_n,s_n}}$
A state $c_{0}$ is initial if for all $i \in [1,n]$,
$l_i = \ell_{i}$ and $\mathbf{V}_{i}$ is the initial valuation of $\bipvar{i}$.
%
A state $c$ is an error if for some $i \in [1,n]$, $l_{i}$ is an error location.
%
We say an interaction $\gamma \in \Gamma$ is enabled on a state $c$ if
 for every component $\comp{i} \in \components$,
 such that $\gamma \cap \ports{i} \neq \emptyset$,
%
 there is an edge $\mktuple{l_i, \gamma \cap \ports{i}, g_i, f_i, l_{i}' } \in \edges{i}$
 and  $\mathbf{V}_{i} \models g_i $.
%
The labeled transition system semantics of a \BIP model is defined as follows.

\begin{definition} [\BIP operational semantics]
\label{operational-semantics}
Given a \BIP model $\bipmodel=\mktuple{\components,\Gamma}$,
 its operational semantics is defined by a labeled transition system
 $\TS{\BIP} = \mktuple{\mathcal{C}, \Sigma, \mathcal{R}, \mathcal{C}_{0}}$, where
%
\begin{enumerate}
\item $\mathcal{C}$ is the set of states,
\item $\Sigma = \Gamma $,
\item $\mathcal{R}$ is the set of transitions,
  and we say that there is a transition from a state $c$
%$c=\mktuple{\mktuple{l_1,s_1}, \ldots, \mktuple{l_n,s_n}} $
  to another state $c'$,
%$c'=\mktuple{\mktuple{l_1',s_1'}, \ldots, \mktuple{l_n',s_n'}}$
  if there is an interaction $\gamma $ such that,
%
\begin{enumerate}
%\begin{inparaenum}[(i)]
  \item $\gamma$ is enabled in $c$;
  \item for all $\comp{i} \in \mathbb{B}$ such that
    $\gamma \cap \ports{i} \neq \emptyset$, there is an edge
    $\mktuple{l_i, \gamma \cap \ports{i}, g_i, f_i, l_i'} \in \edges{i} $,
    %and if $f_i = x := e $,
    then $\mathbf{V}'_{i} = \mathbf{V}_{i}[\bipvar{} / f_i(\bipvar{})]$;
  \item for all $\comp{i} \in \mathbb{B} $ such that
    $\gamma \cap \ports{i} = \emptyset $,
    $l_i' = l_i$ and $\mathbf{V}_i' = \mathbf{V}_i$.
%\end{inparaenum}
\end{enumerate}
\item $\mathcal{C}_{0}$ is the set of initial states.
\end{enumerate}
\end{definition}

%We denote by $c \overset{\gamma} \rightarrow c'$ that
%there is a transition from state $c$ to state $c'$, following the interaction $\gamma$.
%
%We denote by $\rho$
%
%A state $c$ % $c=\mktuple{\mktuple{l_1,s_1}, \ldots, \mktuple{l_n,s_n}}$
%is reachable if there is a trace
%$c_0, c_1, \ldots, c_k$ such that
%$c_0 \xrightarrow{\gamma_1} c_1 \xrightarrow{\gamma_2} \ldots \xrightarrow{\gamma_k} c $,
%from an initial configuration $c_{0}$ to $ c$.
%
In this paper, we do not use temporal logics to specify safety properties,
 but recognize a set of locations as error locations.
 A \BIP model is safe if no error states are reachable.
 Notice that any safety property can be encoded as a reachability
 problem by necessarily creating additional components.

%%%%%%%%%%%%%%%%%%%%%%%%%%%%%%%%%%%%%%%%%%%%%%%%%%%%%%%%%%%%%%%%%%%%%%%%%%%%%%%%%%%%%%%%
%%%%%%%%%%%%%%%%%%%%%%%%%%%%%%%%%%%%%%%%%%%%%%%%%%%%%%%%%%%%%%%%%%%%%%%%%%%%%%%%%%%%%%%%

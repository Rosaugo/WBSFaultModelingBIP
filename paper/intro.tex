
\section{Introduction}
\label{introduction}


Design, manufacture and verification of large scale reliable hardware/software
    systems (e.g. cyber-physical systems) remains a grand challenge in system
    design automation~\cite{sifakis2015}.
To address this challenge, the rigorous system design
    methodology~\cite{sifakis13} and the Behavior-Interaction-Priority (BIP)
    framework~\cite{bip11} have been recently proposed.
%
Rigorous system design can be understood as a formal,
 accountable and coherent process for
 deriving correct and trustworthy system implementations
 from high-level specifications.
%
The essential safety properties of the design are guaranteed
 at the earliest possible design phase
 by applying algorithmic verification to the system model,
 and then the system implementation is automatically generated
 by a sequence of property preserving model transformations,
 progressively refining the model with details specific to the target platforms.


BIP is a component-based system design framework,
 coming with a formal language with well-defined semantics
 and a tool chain supporting rigorous system design process.
 The BIP language offers a three-layered modeling mechanism for
  constructing complex system behavior and architectures~\cite{concur16},
  i.e.,  Behavior, Interaction, and Priority.
%
Behavior is characterized by a set of components,
 which are formally defined as automata extended with linear arithmetic.
%
Interaction specifies the multiparty synchronization of components,
 among which data transfer may take place.
%
Priority can be used to schedule the interactions or resolve conflicts
 when several interactions are enabled simultaneously.
%
The key insight underlying this three-layered modeling mechanism is
 the principle of separation of concerns,
 that is, system computation is captured by a set of components,
 and system coordination is modeled by interaction and priority.
%
Moreover, this layered modeling mechanism also benefits
 the formal verification from allowing us to handle the computation and coordination separately.
%
The BIP tool chain supports both algorithmic verification and testing of high-level system designs
 ~\cite{dfinder10,atva15,tgc15}
 and automatic system synthesis of low-level implementations from high-level system designs~\cite{bip-emsoft10}.
 In practice, BIP has been actively used in several applications~\cite{bipapplication12a,bipapplication18}.


The rest of the paper is structured as follows.
%
In Section~\ref{sec:preliminary} and Section~\ref{sec:bip},
 we introduce the preliminaries and the BIP modeling language.
%
In Section~\ref{sec:relatework} and Section~\ref{sec:conclusions},
 we review the most related works and draw some conclusions and outline directions for future work.


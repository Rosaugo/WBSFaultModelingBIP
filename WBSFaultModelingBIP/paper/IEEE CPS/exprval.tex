

\section{Verification Methodology and Experiments}

In this section, we introduce the application of SBIP tool chains on our WBS BIP model. . Through the experiment results, we demonstrate that our methodology can help system engineers improve the role of specification and promote the iteration of the system architecture.

\subsection{WBS Requirements formalization and decomposition}
The AIR6110 document contains several requirements for the WBS. These can be grouped in two main categories: Requirements corresponding to safety, e.g., \emph{the loss of all wheel braking shall be extremely remote}, and others, e.g., \emph{the WBS shall have at least two hydraulic pressure sources}.

\textcolor{red}{
Giving an explaination of how to translate WBS requirements to LTL specifications and taking one requirement to LTL as an example.
}

\subsection{Experiments and results}
\textcolor{red}{
Briefly introduction.
}

\subsubsection{Probability Estimation}
\
\newline
\textcolor{red}{
\indent Will be texted after further experiments finished.
}

\subsubsection{Parametric exploration}
\
\newline
\textcolor{red}{
	\indent Will be texted after further experiments finished.
}

\section{Related Work}

Model Based Safety Analysis (MBSA) \cite{mbsa05} aims to generate fault injected models for safety analysis. 
 In \cite{mbsa11}, MBSA has been distinguished into two categories according to how the faulty models are constructed.
 The first category performs MBSA on the extension of the design models used in development process with failure mode models.
 In this line of work, the ESACS and ISAAC project develops Extended System Model and Failure Injection methods \cite{erts06,esrel03}. 
 This approach maximizes the consistency between system development and safety analysis. 
 However, a complete extended system may be intractable for analysis tools. 
 The second category creates dedicated models that are specifically for the goal of safety assessment
 Engineers can avoid unnecessary complexity by adjusting the level of detail. 
 Modelling tools such as HiP-HOPS and AltaRica are used to build models for failure analysis specifically following this approach. 

A number of tools have been extended to support both kinds of MBSA approaches, 
 e.g. Architecture Analysis and Design Language (AADL) \cite{cmu07,sae11} 
 and the integration of HiP-HOPS with Matlab Simulink \cite{dsn01}. 
%
 In \cite{aadl13} a dialect of AADL has been proposed to validate failure injected model using a satellite system as case study. 
 With the development of the revised Error Model (EMV2) Annex \cite{aadlemv2} for AADL, 
 engineers have applied AADL on different systems (e.g., Flight Control \cite{field1}, Fire Alarm \cite{field2} and other cyber-physical systems \cite{field3}) 
 to capture key features of failure and error propagations from system models. 
%
 In \cite{aadl_altarica} a translator from AADL to AltaRica has been proposed and applied to a flight control system.
 In \cite{aadl_hiphop}, a transformation from AADL model to HiP-HOPS model has been studied. 
 However, no tool has demonstrated absolute advantages over other tools in performing MBSA.


\section{Introduction}

There is a growing trend to advocate the methodology of model based system design (MBSD) and model based safety analysis (MBSA),
 as a response to the strong demand for minimizing errors during the development of safety critical systems such as aircraft and aerospace systems \cite{lctes10,issrew12}.
 Generally speaking,  MBSD refers to the type of approach 
 that uses an abstract system model to describe the nominal behavior and the system architecture involving both software and hardware components.
 When the system model is built with a formal modelling language, various system level analysis and verification can then be performed on this formal model to ensure the correctness of the design.
 Moreover, low level implementations can also be automatically derived from the design models through code generation.
 MBSA moves one step further to take into account the faulty behavior of software and hardware components,
 and then augments the nominal system model with fault models.
 This extended system model can then be used to study the safety and reliability properties when in presence of one or more faults.
% 
 In practice, MBSD and MBSA has been highly recommended in the SAE (Society of Automotive Engineers) guidance for the development of complex aerospace systems 
  (i.e., Aerospace Recommended Practice ARP4754A \cite{arp4754a} and ARP4761 \cite{arp4761}).
  Particularly, SAE provides an aircraft wheel brake system (WBS) as an example of applying ARP4754A and ARP4761 into practice 
  in Aerospace Information Report AIR6110 \cite{air6110}.


BIP (Behavior-Interaction-Priority) \cite{bip1,bip2} is a component based framework that advocates the methodology of rigorous system design.
 Rigorous system design can be understood as the process that derives correct-by-construction implementations from high level system models automatically.
 BIP provides a formal modelling language that is expressive enough for describing both the functional behavior and the architecture of complex systems.
 It also provides a set of tools that automates the rigorous system design process.
 SBIP \cite{sbip18} is a statistical extension of BIP that supports formal modeling and statistical analysis of systems exhibiting stochastic behaviors.
 Currently, both BIP and SBIP are widely used in the modelling and analysis of nominal system models (i.e., system models withou faulty behaviors).
% In this work, we apply the BIP and SBIP framework for model based safety analysis.

In this work, we propose to apply the BIP framework for fault modelling and safety analysis 
 and perform a case study on the aircraft wheel brake system provided in SAE AIR6110 \cite{air6110}.
 In a related work \cite{cav15}, WBS has been also investigated as a case study to promote their approach to model based safety analysis.
 However,  in their approach fault models are injected to the formal system model in a monolithic manner 
 such that the integration process can obscure the nominal behavior of the system and creats additional overheads for safety analysis.
 Our fault mdoelling is modular and features a clear seperation between nominal behavior and fault behavior that makes the WBS model extendable and understandable.
 We also adopt the statistical approach to model checking the extended system model that uses less time and memory.
% The paper proposes a formal modelling and analysis of WBS based on an integration of several approaches including OCRA, NuXmv and xSAP.
% They using a cluster of machines with massive memories and time, but also faces with time out problems during computation. 
%
 To this end, the main contributions of this work can be summarized as followings.
%
\begin{enumerate}
\item For each reference system architecture of the WBS, we have built a formal model in BIP that is amenable to various formal safety analysis and verification.
 We also propose a method to derive fault models from fault trees, and integrate nominal system model with fault models. 
 The extended system model is a stochastic system where faulty behaviors and failure rate are both manifested in the model.

\item We propose to use the Statistical Model Checking\cite{vmcai04,cav04,cmu04} approach to simulate the failure injected stochastic WBS model,
 and automatically verify whether the BIP system model satisfies the AIR6110 requirements using the statistical model checker SBIP for BIP \cite{sbip18}.
 We report interesting findings and confirm the necessity of the iteration in AIR6110 in the experiments of comparing several WBS architectures. 
 We also observe the performance of SBIP during probability estimation and give some suggestions for using BIP and SBIP tool chains.
\end{enumerate}

The remainder of the paper is organized as follows. 
Section 2 presents the related work on model based safety analysis.
Section 3 introduces the BIP framework and statistical model checking for BIP.
Section 4 presents an overview of the SAE AIR6110 Wheel Brake System.
In section 5, we present the integration process of nominal system model and faulty behavior.
Section 6 provides the verification methods and experimental results.
Section 7 concludes this paper and discuss some future works.

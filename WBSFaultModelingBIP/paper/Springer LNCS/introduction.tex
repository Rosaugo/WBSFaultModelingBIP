\section{Introduction}
Model based design is becoming an increasingly important approach for the development of safety critical systems.
The functional behavior and the architecture of the system can be modelled in some formal language using graphical or textual representations.
This formal model is called the nominal system model and it can be subjected to various analysis such as simulation, verification and testing.
Code generation is also supported to derive implementations directly from high level models.
In model based safety analysis, the nominal system model is then augmented with fault models for the digital and mechanical systems to create an extended system model.
This model can be used to describe the behavior of the system in the presence of one or more faults.

%For the extended model, we suggest \emph{stochastic abstraction}\cite{stoabs} technique to import probability distributions into some of the components having stochastic process transiting between nominal behavior and faulty behavior.
%To compute the probability for the system to satisfy the specification, the Statistical Model Checking\cite{vmcai04}\cite{cav04}\cite{cmu04} approach is an efficiently choice. The SMC approach simulates the system and use the statistic results to verify the given properties.
%This is an alternative way to avoid the state explosion problem, especially for complex real-world systems.

On the other hand, there is a continuously strong demand for minimizing errors during the development of safety-critical complex systems\cite{lctes10}, especially for aerospace systems\cite{issrew12}.
The Society of Automotive Engineers(SAE) provides Aerospace Recommended Practice ARP4754A\cite{arp4754a} and ARP4761\cite{arp4761} as guidances for the developing complex aerospace systems. As a sample of applying ARP4754A and ARP4761 into practice, SAE provides Aerospace Information Report AIR6110\cite{air6110}, which introduces a detailed example of the aircraft and systems development for wheel brake function.

In the past few years, a few researches use WBS as a case study to propose their promotion on model-based safety analysis.
Some of them find it hard to add faults to existing components, and the integration process can obscure the nominal behaviors of the model.  The paper\cite{cav15} proposes a formal modeling and analysis of WBS based on an integration of serveral approaches including OCRA, NuXmv and xSAP. They using a cluster of machines with massive memories and time, but also faces with time out problems during computation. It is desired to propose verification method using less time and memory on an extendable and understandable WBS model.

\emph{Main contribution:} 
In this paper, we propose to apply the stochastic abstraction approach on Wheel Brake System(WBS) model extracted from the SAE AIR6110 standard. The WBS is designed into several BIP models with four architecures. Our contribution is threefolds:
%BIP is a framework, language and tool for specifying components and component interactions. BIP performs well in generating executions of hierarchical components based systems. BIP also provides a type definition for each component and graphical representation for BIP model to decrease the complexity of system design.

1. BIP Model of the WBS. We design several BIP models with four architectures extracted from the SAE AIR6110 standard. BIP is a tool for specifying components and component interactions. One of the very attractive features of BIP is its ability to generate executions of composite systems. Another advantage of BIP is that it permits to give a very detailed description of each component. BIP also provides graphical and texture representation for engineers to decrease the complexity of the design.

2.Extended BIP model for MBSA. We introduce a method to derive fault models from WBS fault trees, and to integrate nominal system model with fault models. The extended model is a stochastic system where faulty behaviors and failure rate are manifested in the model.
 
3. Verification method. Since traditional verification methods face state explosion problems, we use the Statistical Model Checking\cite{vmcai04}\cite{cav04}\cite{cmu04} approach to simulate the failure injected stochastic WBS model. Using results from the statistic area, we can decide if the failure has successfully injected into the model and, more importantly, whether the system satisfy the AIR6110 requirements or not.

We complete our modeling and verification process using SBIP\cite{sbip18}, a statistical model checker for BIP models.
We report interesting findings when taking the experiment results of serveral WBS architectures into comparison. As another promotion of experiments, we confirm the necessity of the iteration in AIR6110. We also observe the performance of SBIP during probability estimation and give some suggestions for using BIP and SBIP tool chains.

\emph{Paper structure.}
The remainder of the paper is organized as follows. Section 2 introduces the BIP framework and statistical model checking for BIP. 
Section 3 presents an overview of the SAE AIR6110 Wheel Brake System. 
In section 4, we present the integration process of nominal system model and faulty behavior. 
Section 5 provides the verification methods and experimental results. 
Section 6 concludes this paper and discuss some future works.
